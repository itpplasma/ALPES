\chapter{Radiation protection}

\section{Plasma radiation}

Accelerated charged particles are sources of electromagnetic radiation.
When particles are accelerated within electric or magnetic fields, they emit radiation with unique properties.
In plasma, different types of radiation may occur. The simplest and most inevitable of these radiation emissions is Bremsstrahlung. This is caused by binary collisions between electrons and ions which produce photons with energy comparable to the plasma temperature \cite{NucFus}.
When external radiation fields interact with plasma, scattered radiation is produced.
Charged particles moving through magnetic fields emit either cyclotron or synchrotron radiation, depending on the energy range of the particles.
Radiation can also be generated by atomic processes in plasmas with impurities, called line and recombination radiation \cite{PhyscPlasma}.
Recombination radiation occurs when a free electron is captured by an ion, resulting in the emission of a photon.
This type of emission is typically in the ultraviolet (UV) to visible range for many plasmas, depending on the energy levels of the ions.
Line emission instead occurs when electrons in ions transition between discrete energy levels, emitting photons at specific wavelengths.
The specific lines emitted depend on the elements present in the plasma.

\section{Radiation shielding}

In general, a radiation shield must effectively balance the following aspects as well as ensure adequate protection from the radiation emissions\cite{RadPhd}:

\begin{enumerate}
      \item \textbf{Activation:}
            An activation product is a material that has been made radioactive by the process of neutron activation.
            Ideally a shield would be made of a material that cannot be activate.
            In practice, using a low-activation material like vanadium would be rather expensive.
            Through careful shield design, the activation of the shield can be significantly reduced.

      \item \textbf{Dose:}
            Radiation dose is a measure of the amount of exposure to radiation.
            During the design and operation of a radiation facility there must be a radiation protection system that ensures that the received doses are well below the prescribed limits.

      \item \textbf{Heating/Cooling:}
            The nuclear heating of the shield should be kept minimal to prevent excessive energy from being deposited within the shield.
            Where this cannot be avoided, cooling methods must be introduced.

      \item \textbf{Weight:}
            Shield weight is not typically a concern in stationary reactor applications.
            However certain parts of reactors may have strict weight limits if moving or maintenance is required.
            For mobile shields, it is crucial that their weight is not excessive to prevent damage to the equipment used for moving them.
            This is particularly important for diagnostic ports.

      \item \textbf{Radiation damage:}
            It is important to consider the level to which a shield will become damaged in the reactor application, and if this damage can not be avoided then shield replacement should be provided.

\end{enumerate}

For our purpose, only dose and cooling are relevant.


\section{Radiation shielding design}
There are a number of stages when designing a radiation shield \cite{RadShi}:

\begin{enumerate}
      \item \textbf{Study of the Primary Radiation Source:}
            The primary radiation source determines the materials and geometry needed for the construction of the shield.
            Knowledge of the spatial, angular and energetic distribution of the source is essential for any future calculation.


      \item \textbf{Formulation of the Basic Shield:}
            Once the radiation type and energy distribution of the source particles is known, the primary shielding materials can be selected depending upon how much attenuation is required.

      \item \textbf{Calculation of the Attenuation of Primary Radiations:}
            On the basis of the over-all system evaluation and the choice of material discussed above, approximate calculations can be performed to obtain the thicknesses required to attenuate the primary radiation.

\end{enumerate}

As explained in the previous section, for our stellarator we will mainly consider Bremsstrahlung radiation, which produce photons with energy comparable to the plasma temperature.
Our plasma parameters are similar to the ones in the TJ-K stellarator, thus we will be working with low-temperature plasma.
Typical plasma parameters for TJ-K are:


\begin{table}[H]
      \centering
      \caption{Parameters of the TJ-K plasma experiment  \cite{TJK}.}
      \begin{tabular}{>{\raggedright\arraybackslash}p{4cm} >{\raggedright\arraybackslash}p{4cm}}
            \toprule
            \textbf{Parameter}   & \textbf{Value}                \\
            \midrule
            Density              & $\SI{5e17}{\per\cubic\meter}$ \\
            Electron temperature & $\SI{10}{\electronvolt}$      \\
            Ion temperature      & $\SI{1}{\electronvolt}$       \\
            Working gases        & H, D, He, Ne, Ar              \\
            \bottomrule
      \end{tabular}
      \label{table:TJ-K_parameters}
\end{table}



\section{Material}

In the selection of materials for the vacuum vessel it is important to choose non-magnetizable substances to avoid interference with the surrounding magnetic fields.
One of the best option meeting this criterion is austenitic stainless steel, specifically grades 304 and 316.
Steel presents many advantages for shielding due to its different properties:

\begin{enumerate}
      \item \textbf{Density:}
            With its high density, steel effectively attenuates ionizing radiation such as gamma rays and X-rays, ensuring robust protection within the vacuum vessel.

      \item \textbf{Availability:}
            Steel is widely accessible and relatively cheaper compared to some other materials used for radiation shielding.

      \item \textbf{Strength:}
            The inherent strength of steel ensures structural integrity, especially in applications where the shielding material also serves as a structural component.

      \item \textbf{Versatility:}
            Steel can be easily shaped and fabricated into various forms, allowing for flexibility in design and construction of shielding structures.

      \item \textbf{Vacuum compatibility:}
            Stainless steel is commonly used in vacuum applications due to its ability to maintain performance and integrity in vacuum environments.

\end{enumerate}


\section{X-ray radiation and shielding}

X-rays radiation will come from Bremsstrahlung radiation, which is primarily dependent on the electron temperature.
Thus the photons will have energies up to $\SI{10}{\electronvolt}$.
The frequency of an electromagnetic wave with an energy of $\SI{10}{\electronvolt}$ is approximately $\SI{2.4e15}{\hertz}$, which lies on the UV spectrum.
In fact, given the plasma parameters of stellarator TJ-K, while the plasma density is relatively high, the electron and ion temperatures are quite low for significant X-ray production.

Although significant X-ray radiation is not expected from the given plasma parameters, it is still important to have some shielding against X-rays, especially if there are any unexpected events or variations in the plasma conditions.
Stainless steel is less effective than materials like lead or tungsten for x-ray shielding.
However, the effectiveness of stainless steel as a shield increases with its thickness.
For low-energy x-rays, a few millimeters of stainless steel might provide adequate shielding.

For a beam of mono-energetic photons, the intensity of photons transmitted across some distance x in a material can be expressed with the equation

\begin{equation}
      I = I_{0} e ^{- \mu x}
\end{equation}

where $I_0$ is the initial intensity of photons and $\mu$ is the linear attenuation coefficient.
The linear attenuation coefficient describes the fraction of a radiation beam that is absorbed or scattered per unit thickness of the absorber.
It it is photon energy dependent\cite{HVL}.

Since the linear attenuation coefficient of steel 316L for $\SI{10}{\electronvolt}$ energy radiation is not provided, only estimations and comparison with already constructed devices can be made.


For the design and construction of the small modular stellarator for magnetic confinement of plasma in Costa Rica (SCR-1), the vessel material was chosen to be aluminum, with an approximated thickness of $\SI{10}{\milli\meter}$. Although using austenitic 304L grade stainless steel was analyzed, it was discarded because of the difficulty to manufacture parts according to the device dimensions and because it increased greatly project costs\cite{SCR}.
The plasma parameters were electron temperature $T_e = \SI{15}{\electronvolt}$ and electron density $n_e = \SI{e17}{\per\cubic\meter}$.

At the National Research Nuclear University MEPhI in Moscow, a small educational and demonstration
spherical tokamak MEPhIST is under construction. The domes of the chamber were made of AISI 321
steel sheets with an initial thickness of $\SI{2.5}{\milli\meter}$. The rest of the chamber elements were manufactured from AISI 316 steel. The thickness of the inner cylinder was $\SI{1}{\milli\meter}$. The thickness of the outer cylinders was $\SI{3}{\milli\meter}$ to facilitate welding with rectangular flanges \cite{MEP}.

TJ-K is a stellarator which has been constructed and operated in Madrid under the name TJ-I U. In
2006, TJ-K was moved to the university of Stuttgart. Since our stellarator will have the same plasma parameters as the TJ-K stellarator, we can design our vacuum chamber in a similar manner.
The vessel is made of stainless steel 316L and has a wall thickness of about $\SI{10}{\milli\meter}$ \cite{TJK2}.



\section{Microwave radiation and shielding}

For heating the plasma we will use microwave heating system operating at frequency of $\SI{2.45}{\giga\hertz}$.
Stainless steel is also a good choice material for microwave radiation shielding due to its high conductivity.

The skin depth $\delta$ describes the ability of electromagnetic radiation (particularly microwave) to penetrate a material\cite{skin}.
The skin depth is related to the resistivity $\rho$, magnetic permeability $\mu$ and angular frequency $\omega$ according to the equation \cite{emrad}

\begin{equation}
      \delta =\sqrt{\frac{2 \rho}{\mu \omega}}
\end{equation}

Stainless steel 316L has resistivity  $\rho = \SI{7.4e-7}{\ohm\meter}$ \cite{cond}.
The relative magnetic permeability $\mu_r$ is close to $1$ (non-magnetic austenitic stainless steel).
The magnetic permeability $\mu = \mu_{r} \cdot \mu_0$ is then $\mu \approx 4 \pi \cdot \SI{e-7}{\henry\per\meter}$, where $\mu_0$ is vacuum magnetic permeability \cite{codata22}.
The skin depth obtained for our frequency is $\delta \approx \SI{8.76}{\micro\meter}$.
To achieve significant attenuation of microwaves, the material thickness should be several skin depths.
Typically, a thickness of about $5$ skin depths is sufficient.
The final thickness $t$ required for efficiently shield from microwave radiations is then $t \approx \SI{43.8}{\micro\meter}$.
Since our vessel will be a few millimeters thick for X-ray shielding and structural integrity, this thickness is more than sufficient for microwave radiation shielding.
In fact, it is orders of magnitude greater than the calculated necessary thickness for microwaves attenuation, providing efficient protection against microwave radiation.


\section{Windows}

For our project, we decide to design the stellarator with windows to look inside the vacuum chamber. Since significant X-ray radiation is not expected from the given plasma parameters, lead glass in not necessary for our purpose.
We decided for quartz glass instead.
For a thin, circular plate with a clamped edge subjected to a uniform pressure difference across its faces, the maximum stress can be estimated as \cite{glass, elast, stress}:

\begin{equation}
      \sigma_{max}=\sqrt{1-\nu+\nu^2} \frac{3}{4} \frac{p r^2}{t^2}
\end{equation}

where $\nu$ is the Poisson's ratio of the glass, p is the pressure difference, t is the thickness, r is the radius of the window.

For safe operation must be valid:

\begin{equation}
      \sigma_{max} \leq \sigma_{t}
\end{equation}

where $\sigma_{t}$ is the tensile strength of the glass.

For $SiO_{2}$ (quartz glass), $\sigma_{t} = \SI{50}{\mega\pascal}$ and $\nu = 0.17$ \cite{quartz}.

For a pressure difference $ p = p_{out}-p_{in} \approx p_{out} = p_{atm} = \SI{101300}{\pascal}$ ($p_{in} \approx \SI{e-8}{\milli\bar} \approx 0$) and a radius $ r = \SI{100}{\milli\meter} $, we obtain a thickness of about $t \geq \SI{3.75}{\milli\meter}$.

For safety reasons, it's important to include a safety factor in the thickness calculation.
A safety factor compensates for uncertainties in material properties, load estimations, and potential flaws in the material.
Safety factors typically range from $1.5$ to $3.0$ depending on the criticality of the application and the material's reliability.
For our purpose, we use a SF of about $2.7$.
Thus, our thickness is $t = \SI{10}{\milli\meter}$.
With this thickness, the maximum stress is  $\sigma_{max} \approx \SI{7}{\mega\pascal}$.
This value is more than $7$ times less than tensile strength of $SiO_{2}$ glasses.

\section{Conclusion}

For our vacuum vessel, we have designed it with a thickness of $\SI{10}{\milli\meter}$, which provides adequate protection against X-ray and microwave radiation.
The chosen material is austenitic stainless steel 316L.
This material was selected due to its non-magnetic nature and its advantageous properties, including high strength, good availability, and overall versatility, making it an ideal choice for our application.

For the window, we designed it with a circular shape having a radius $\SI{100}{\milli\meter}$ made of quartz glass.
The ideal thickness for the window is $\SI{10}{\milli\meter}$.

\section{Outlook}

A $\SI{10}{\milli\meter}$ thickness of stainless steel 316L should generally be robust enough to withstand the vacuum pressure.
However, it is important to perform a detailed structural analysis to ensure the chamber can withstand external atmospheric pressure and any mechanical stresses without significant deformation or failure.
Additionally, the window's mounting and sealing must be carefully designed to prevent leaks and ensure stability.
It is also necessary to ensure that the design meets relevant safety standards and regulations for vacuum systems and radiation protection.
Safety features such as emergency shutdown systems and radiation monitoring should be included.
For this reason, integration of various sensors and instrumentation for monitoring temperature, radiation levels and other critical parameters should be considered.
Moreover, it is essential to develop safety protocols for handling and maintenance, ensuring personnel are aware of and protected from potential radiation exposure.
