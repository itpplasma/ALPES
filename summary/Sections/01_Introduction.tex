\section{Introduction}
The following report documents the progress and intermediate results of "Fusion Reactor Design", a joint course between Graz University of Technology and the Technical University of Munich, supported by Proxima Fusion. It consists of a student project where the goal is to design a fusion reactor with stellarator geometry, with priority given to the following properties:
\begin{itemize}
    \item \textbf{Size}: The reactor should be able to fit through a small door, imposing an upper size limit of (190x90)~cm.
    \item \textbf{Aspect ratio and plasma volume:} While there is no hard limit here, the stellarator should have a high plasma volume, and thus the aspect ratio should be as low as possible without compromising other properties (stability, alpha particle losses, island reduction).
    \item \textbf{Coil simplicity:} The coils should be possible for a group of students to understand and theoretically construct. Realistically, this means that the number of coils and, more importantly, the number of coil types, should be kept at a minimum.

\end{itemize}
To realize this, the participants were assigned to five groups corresponding to different aspects of the reactor, depending on areas of expertise. These aspects are: 1) Coils, 2) Design and Optimization, 3) Diagnostics, 4) Heating, and 5) Vacuum. The exact progress and results of each of the groups are presented in the dedicated chapters, with brief summaries given here.\\
\\
The Coils team is responsible for the design of the magnetic field coils given a configuration by the design team. Here, the main concerns lie in the coil materials, insulation and design, all while retaining proper spacing such that the configuration can fit into the vacuum chamber.\\
\\
The Design and Optimization team provides the first coil configuration and optimizes parameters such as alpha losses and the iota profile, with the goal of stability against coil construction errors and deformations. Additionally, a 3D printed model is designed.\\
\\
The Diagnostics team researches and proposes configurations of various measurement methods for verification of the magnetic field inside the reactor. Here, the methods include a) interferometry, b) Langmuir probe and c) Rogowski coil and d) diamagnetic coil.\\
\\
The Heating team examines possible heating systems, specifically 2.45 GHz microwave technology, based on costs and whether it fits into the reactor. This is done by analyzing temperature and frequency profiles (the latter for both underdense and overdense plasma).\\
\\
The Vacuum team is tasked with designing a functional vacuum chamber satisfying the size requirements while being able to contain all aforementioned parts. It should be capable of withstanding ultra high vacuum pressures, radiation and elements required to create the plasma.\\
\\
Over the course of the semester, these teams have worked together to achieve a basic concept for a full design, which has potential to be built given proper time and funding.