\section{Conclusion}
In 2024, the joint course \textit{Fusion Reactor Design} was held for the first time in a partership of TU Graz, TU Munich and Proxima Fusion GmbH.
\\
The Coils, Design, Diagnostics, Heating and Vacuum teams have each contributed and worked together to design and provide a concept for a small stellarator magnetic plasma confinement experiment in each of their respective fields of expertise. The requirements decided at the beginning of the project were prioritized and are conceptually fulfilled: 1) Size limit of 190x90 cm, 2) low aspect ratio of around 4, and 3) few (three) different coil types. As a final result, a CAD model with all the components contributed by the various teams is available.\\
\\
As with any project of this complexity, there are various areas of possible improvement. On the theoretical end, more simulations to verify the coil structure, as well as continued optimization to improve general design stability and lowering the aspect ratio, could be performed. Additionally, refinement of the diagnostics circuits, study of the heating design for the complexities of a stellarator and structural and safety analysis of the vacuum chamber are further possibilities.\\
\\
Because the practical/experimental portion had lower priority in this project, it is essentially open end. The specific materials and suppliers for construction of the coils and vacuum chamber have been considered, but not finalized, and of course there is 1) the actual construction and 2) the experimental verification of the functionality of the entire design. Given the funding and time (possible future semesters), these would be highly interesting future continuation options for the project.

\section*{Acknowledgements} This work has been carried out within the framework of the EUROfusion Consortium, funded by the European Union via the Euratom Research and Training Programme (Grant Agreement No 101052200 — EUROfusion). Views and opinions expressed are however those of the author(s) only and do not necessarily reflect those of the European Union or the European Commission. Neither the European Union nor the European Commission can be held responsible for them. We thank Max-Planck-Institute for Plasma Physics, in particular the team of Eve Stenson. We gratefully acknowledge support from NAWI Graz, Proxima Fusion, and TU Graz via the seed funding and Joint Course initiatives.